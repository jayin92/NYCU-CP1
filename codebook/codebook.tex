\documentclass[8pt, twocolumn]{extarticle}
\usepackage{fontspec}   %加這個就可以設定字體
\usepackage{xeCJK}       %讓中英文字體分開設置
\usepackage{indentfirst}
\usepackage{listings}
\usepackage[newfloat]{minted}
\usepackage{float}
\usepackage{graphicx}
\usepackage{caption}
\usepackage{fancyhdr}
\usepackage{hyperref}
\usepackage{amsmath}
\usepackage{multirow}
\usepackage[dvipsnames]{xcolor}
\usepackage{graphicx}
\usepackage{tabularx}
\usepackage{booktabs}
\usepackage{caption}
\usepackage{subcaption}
\usepackage{pifont}
\usepackage{amssymb}
\usepackage{upquote}
\usepackage{titling}
\usepackage{multicol} % added package



\usepackage[breakable, listings, skins, minted]{tcolorbox}
\usepackage{etoolbox}
\renewtcblisting{minted}{%
	listing engine=minted,
	minted language=cpp,
	listing only,
	breakable,
	enhanced,
	minted options = {
		linenos, 
		breaklines=true, 
		breakbefore=., 
		fontsize=\footnotesize, 
		numbersep=2mm
	},
	overlay={%
		\begin{tcbclipinterior}
			\fill[gray!25] (frame.south west) rectangle ([xshift=4mm]frame.north west);
		\end{tcbclipinterior}
	}   
}

\usepackage[
top=1.5cm,
bottom=1.5cm,
left=2.5cm,
right=2.5cm,
includehead,includefoot,
heightrounded, % to avoid spurious underfull messages
]{geometry} 



\newenvironment{code}{\captionsetup{type=listing}}{}
\SetupFloatingEnvironment{listing}{name=Code}
\usepackage[extreme]{savetrees}

\title{Codebook for CP1 Final Exam}
\author{110550088 李杰穎}
\date{}

\setCJKmainfont{Noto Serif TC}
\setmonofont[Mapping=tex-text]{Consolas}

\XeTeXlinebreaklocale "zh"             %這兩行一定要加,中文才能自動換行
\XeTeXlinebreakskip = 0pt plus 1pt     %這兩行一定要加,中文才能自動換行

\setlength{\parindent}{2em}
\setlength{\parskip}{0.5em}
\renewcommand{\baselinestretch}{1}
\setlength{\droptitle}{-10em}   % This is your set screw


\begin{document}
\maketitle
\section{Template}
\begin{code}
\begin{minted}
// :80 <enter>
#include <bits/stdc++.h>
using namespace std;
typedef long long ll;
typedef pair<int, int> pii;
typedef pair<ll, ll> pll;
typedef pair<int, ll> pil;
typedef pair<ll, int> pli;
typedef pair<double,double> pdd;
#define SQ(i) ((i)*(i))
#define MEM(a, b) memset(a, (b), sizeof(a))
#define SZ(i) int(i.size())
#define FOR(i, j, k, in) for (int i=j ; i<(k) ; i+=in)
#define RFOR(i, j, k, in) for (int i=j ; i>=(k) ; i-=in)
#define REP(i, j) FOR(i, 0, j, 1)
#define REP1(i,j) FOR(i, 1, j+1, 1)
#define RREP(i, j) RFOR(i, j, 0, 1)
#define ALL(_a) _a.begin(),_a.end()
#define mp make_pair
#define pb push_back
#define eb emplace_back
#define X first
#define Y second
#ifdef jayinnn
#define TIME(i) Timer i(#i)
#define debug(...) do{\
    fprintf(stderr,"%s - %d (%s) = ",__PRETTY_FUNCTION__,__LINE__,#__VA_ARGS__);\
    _do(__VA_ARGS__);\
}while(0)
template<typename T>void _do(T &&_x){cerr<<_x<<endl;}
template<typename T,typename ...S> void _do(T &&_x,S &&..._t){cerr<<_x<<", ";_do(_t...);}
template<typename _a,typename _b> ostream& operator << (ostream &_s,const pair<_a,_b> &_p){return _s<<"("<<_p.X<<","<<_p.Y<<")";}
template<typename It> ostream& _OUTC(ostream &_s,It _ita,It _itb)
{
    _s<<"{";
    for(It _it=_ita;_it!=_itb;_it++)
    {
        _s<<(_it==_ita?"":",")<<*_it;
    }
    _s<<"}";
    return _s;
}
template<typename _a> ostream &operator << (ostream &_s,vector<_a> &_c){return _OUTC(_s,ALL(_c));}
template<typename _a> ostream &operator << (ostream &_s,set<_a> &_c){return _OUTC(_s,ALL(_c));}
template<typename _a> ostream &operator << (ostream &_s,deque<_a> &_c){return _OUTC(_s,ALL(_c));}
template<typename _a,typename _b> ostream &operator << (ostream &_s,map<_a,_b> &_c){return _OUTC(_s,ALL(_c));}
template<typename _t> void pary(_t _a,_t _b){_OUTC(cerr,_a,_b);cerr<<endl;}
#define IOS()
class Timer {
private:
    string scope_name;
    chrono::high_resolution_clock::time_point start_time;
public:
    Timer (string name) : scope_name(name) {
        start_time = chrono::high_resolution_clock::now();
    }
    ~Timer () {
        auto stop_time = chrono::high_resolution_clock::now();
        auto length = chrono::duration_cast<chrono::microseconds>(stop_time - start_time).count();
        double mlength = double(length) * 0.001;
        debug(scope_name, mlength);
    }
};
#else
#define TIME(i)
#define debug(...)
#define pary(...)
#define endl '\n'
#define IOS() ios_base::sync_with_stdio(0);cin.tie(0)
#endif

const ll MOD = 1000000007;
const ll INF = 0x3f3f3f3f3f3f3f3f;
const int iNF = 0x3f3f3f3f;
const ll MAXN = 100005;

mt19937 rng(chrono::steady_clock::now().time_since_epoch().count());

void solve(){
    
}

/********** Good Luck :) **********/
int main () {
    TIME(main);
    IOS();
    int t = 1;
    cin >> t;
    while(t--){
        solve();
    }

    return 0;
}
\end{minted}
\end{code}

\section{BIT}
\begin{code}
\begin{minted}
class BIT{
private:
    vector<ll> bit;
public:
    BIT(): bit(vector<ll>(MAXN, 0)) {}
    ll lowbit(ll x){
        return x & (-x);
    }

    ll query(ll idx){
        if(idx == 0) return 0;
        ll ans = 0;
        for(; idx>0; idx-=lowbit(idx)){
            ans += bit[idx];
        }

        return ans;
    }

    void modify(ll idx, int val){
        for(; idx <= n; idx += lowbit(idx)){
            bit[idx] += val;
        }
    }
};
\end{minted}
\end{code}



\section{Dijkstra}
\begin{code}
\begin{minted}
// djisktra with priority queue and memory optimize

#include <bits/stdc++.h>
using namespace std;
typedef long long ll;
typedef pair<ll, ll> pll;
#define pb push_back
#define mp make_pair


const ll INF = 0x3f3f3f3f3f3f3f3f;

int n, m;
vector<vector<pll>> w;
vector<ll> d;
vector<int> parent;

void dijkstra(int src){
    // vector<bool> visit(n+1, false);
    priority_queue<pll, vector<pll>, greater<pll>> pq; // first: weight, second: vertex
    for(int i=1;i<n;i++){
        pq.push(mp(INF, i));
    }
    d[src] = 0;
    parent[src] = src;
    while(!pq.empty()){
        pll edge = pq.top();
        pq.pop();
        ll u = edge.second;
        for(auto i:w[u]){
            ll w = i.second;
            ll v = i.first;
            ll alt = d[u] + i.second;
            if(alt < d[v]){
                d[v] = alt;
                parent[v] = u;
                pq.push(mp(alt, v));
            }
        }
    }
}

int main () {

    cin >> n >> m;
    w.resize(n+1);
    d.resize(n+1, INF);
    parent.resize(n+1, -1);
    ll a, b, tmp;
    for(int i=0;i<m;i++){
        cin >> a >> b >> tmp;
        w[a].pb(mp(b, tmp));
        w[b].pb(mp(a, tmp));
    }
    dijkstra(1); // source = 1
    vector<int> ans;
    int cur;
    if(d[n] != INF){ // if d[n] = INF, there is no shortest path from vertex 1 to vertex n
        cur = n;
        while(cur != 1){
            ans.push_back(cur);
            cur = parent[cur];
        }
        ans.push_back(1);
        int sz = ans.size();
        for(int i=sz-1;i>=0;i--){
            cout << ans[i] << " ";
        }
        cout << endl;
    } else {
        cout << -1 << endl;
    }
    

    return 0;
}
\end{minted}
\end{code}
\section{Segment Tree}
\begin{code}
\begin{minted}
// 利用線段樹解決區間求和 & 單點修改
#include <bits/stdc++.h>

using namespace std;

struct Node{
    int val;
    Node *lc, *rc;
    void pull(){
        val = lc -> val + rc -> val;
    }
};

const int n = 5;

int v[n + 1] = {0, 1, 2, 3, 4, 5};

Node* build(int L, int R){
    Node *node = new Node();
    if(L == R){
        node -> val = v[L];
    }
    int mid = (L+R) >> 1;
    node -> lc = build(L, mid);
    node -> rc = build(mid + 1, R);
    node -> pull();

    return node;
}

void modify(Node* node, int L, int R, int i, int d){
    if(L == R){
        assert(L == i);
        node->val += d;
        return;
    }
    int mid = (L + R) >> 1;
    if(i  <= mid){
        modify(node -> lc, L, mid, i, d);
    } else {
        modify(node -> rc, mid + 1, R, i ,d);
    }
    node -> pull();
}

int query(Node* node, int L, int R, int ql, int qr){
    if(ql > R || qr < L) return 0;
    if(ql <= L &&  R <= qr) return node -> val;
    int mid = (L + R) >> 1;
    return query(node -> lc, L, mid, ql, qr) +
           query(node -> rc, mid + 1, R, ql, qr);
}
\end{minted}
\end{code}
\section{Segment Tree w/ lazy tags}
\begin{code}
\begin{minted}
#include <bits/stdc++.h>
using namespace std;
typedef long long ll;
typedef pair<int, int> pii;
typedef pair<ll, ll> pll;
typedef pair<int, ll> pil;
typedef pair<ll, int> pli;
typedef pair<double,double> pdd;

struct Node{
    int val, tag;
    Node *lc, *rc;
    Node(){
        tag = val = 0;
        lc = rc = nullptr;
    }
    void pull(){
        val = lc -> val + rc -> val;
    }
};

const int n = 5;
int v[n+1] = {0, 1, 16, 2, 8, 4}; // input, 1-base

Node* build(int L, int R){
    Node *node = new Node();
    if( L == R ){
        node -> val = v[L];
        return node;
    }
    int mid = (L + R) >> 1;
    node -> lc = build(L, mid);
    node -> rc = build(mid+1, R);
    node -> pull();
    return node;
}

void push(Node* node, int L, int R){
    if(!node -> tag) return;
    if(L != R){
        int mid = ( L + R ) >> 1;
        node -> lc -> tag += node -> tag;
        node -> rc -> tag += node -> tag;
        node -> lc -> val += node -> tag * (mid - L + 1); 
        node -> rc -> val += node -> tag * ( R - mid ); 
    }
    node -> tag = 0;
}

void modify(Node* node, int L, int R, int ql, int qr, int d){
    debug(L, R);
    if(ql > R || qr < L) return;
    if(ql <= L && R <= qr){
        node -> tag += d;
        node -> val += d * (R - L + 1);
        return;
    }
    push(node, L, R);
    int mid = (L + R) >> 1;
    modify(node -> lc, L, mid, ql, qr, d);
    modify(node -> rc, mid+1, R, ql, qr, d);
    node -> pull();
}

int query(Node* node, int L, int R, int ql, int qr){
    if(ql > R || qr < L) return 0;
    if(ql <= L && R <= qr) return node -> val;
    push(node, L, R);
    int mid = (L + R) >> 1;
    return query(node -> lc, L, mid, ql, qr) + query(node -> rc, mid + 1, R, ql, qr);
}
/********** Good Luck :) **********/
int main () {
    TIME(main);
    IOS();
    Node* root = build(1, n);
    // cout << query(root, 1, n, 1, 5) << endl;
    modify(root, 1, n, 1, 4, 3);

    // cout << query(root, 1, n, 3, 3) << endl;

    return 0;
}
\end{minted}
\end{code}
\section{Disjoint Set}
\begin{code}
\begin{minted}
#include <bits/stdc++.h>
using namespace std;

const int n = 100;

vector<int> p(n);
vector<int> sz(n, 1);

void init(){
    for(int i=0;i<n;i++){
        p[i] = i;
    }
}

int find(int x){
    if(p[x] == x) return x;
    
    return p[x] = find(p[x]);
}

void merge(int x, int y){
    int fx = find(x);
    int fy = find(y);
    if(fx == fy) return;
    if(sz[fx] < sz[fy]) swap(fx, fy);
    
    p[fy] = fx;
    sz[fx] += sz[fy];
}

int main(){
    
}
\end{minted}
\end{code}
\section{Fast Power MOD}
\begin{code}
\begin{minted}
// C++ program to find 
// (a^b)%m for b very large. 
#include <bits/stdc++.h> 
#define ll long long int 
using namespace std; 

// Function to find power 
ll power(ll x, ll y, ll p) 
{ 
    ll res = 1; // Initialize result 

    // Update x if it is more than or 
    // equal to p 
    x = x % p; 

    while (y > 0) { 
        // If y is odd, multiply x with the result 
        if (y & 1) 
            res = (res * x) % p; 

        // y must be even now 
        y = y >> 1; // y = y/2 
        x = (x * x) % p; 
    } 
    return res; 
} 
// Driver Code 
int main() 
{ 
    ll a = 3; 

    // String input as b is very large 
    string b = "100000000000000000000000000"; 

    ll remainderB = 0; 
    ll MOD = 1000000007; 

    // Reduce the number B to a small number 
    // using Fermat Little 
    for (int i = 0; i < b.length(); i++) 
        remainderB = (remainderB * 10 + b[i] - '0') % (MOD - 1); 

    cout << power(a, remainderB, MOD) << endl; 
    return 0; 
} 
\end{minted}
\end{code}
\section{Kosaraju}
\begin{code}
\begin{minted}
// same scc will number 

#include <bits/stdc++.h>

using namespace std;

const int N = 100; // # of vertex

vector<vector<int>> g; // graph
vector<vector<int>> r; // reversed graph
vector<int> order;
vector<int> scc(N);
vector<bool> vis(N);

void RevDfs(int cur){
    vis[cur] = true;
    for(int u: r[cur]){
        if(!vis[u]){
            RevDfs(u);
        }
    }
    order.push_back(cur); // topological order
}

void Dfs(int cur, int s){
    scc[cur] = s;
    for(int u: g[cur]){
        if(scc[u] == -1) Dfs(u, s);
    }
}

void Kosaraju(int n){
    fill(vis.begin(), vis.end(), false);
    fill(scc.begin(), scc.end(), -1);

    for(int i=0;i<n;i++){
        if(!vis[i]) RevDfs(i);
    }

    int n_scc = 0;
    for(int i=n-1;i>=0;i--){
        int cur = order[i];
        if(scc[cur] == -1){
            Dfs(cur, n_scc);
            n_scc++;
        }
    }
}
\end{minted}
\end{code}
\section{Treap}
\begin{code}
\begin{minted}
#include <cstdio>
#include <algorithm>
#include <stack>
#include <ctime>
#include <cstdlib>
#include <queue>
#define MAXN 800000
#define INF 2147483647
using namespace std;
struct treap
{
    int v;
    int sz;
    int p;
    int mn;
    int rev;
    int add;
    treap *l, *r;
    treap() {}
    treap(int k) : v(k), sz(1), p(rand()), mn(k), rev(0), add(0), l(NULL), r(NULL) {}
};

treap mempool[MAXN];
treap* ptr;
treap* gc; // use treap as linked list to garbage collect

inline void init() {
    ptr = mempool;
    gc = NULL;
}

inline void Del(treap* t) {
    t->l = gc;
    gc = t;
}

inline treap* New(int v) {
    if (gc == NULL) {
        *ptr = treap(v);
        return ptr++;
    } else {
        treap* t = gc;
        gc = gc->l;
        *t = treap(v);
        return t;
    }
}

inline int size(treap* t) {
    return t != NULL ? t->sz : 0;
}

inline int small(treap* t) {
    return t != NULL ? t->mn + t->add : INF;
}

inline void pull(treap* t) {
    if (t == NULL)
        return;
    t->sz = 1 + size(t->l) + size(t->r);
    t->mn = min(t->v, min(small(t->l), small(t->r)));
}

inline void reverse(treap* t) {
    if (t != NULL)
        t->rev ^= 1;
}

inline void addn(treap* t, int v) {
    if (t != NULL)
        t->add += v;
}

inline treap* push(treap* t) {
    if (t != NULL) {
        if (t->rev) {
            swap(t->l, t->r);
            reverse(t->l);
            reverse(t->r);
            t->rev = 0;
        }

        if (t->add) {
            t->v += t->add;
            t->mn += t->add;
            addn(t->l, t->add);
            addn(t->r, t->add);
            t->add = 0;
        }
    }
    return t;
}

void split(treap* t, int k, treap*& a, treap*& b) { // split first k nodes from t to a, others to b
    push(t);

    if (t == NULL) {
        a = b = NULL;
    } else if (size(t->l) + 1 <= k) {
        a = t;
        split(t->r, k - size(t->l) - 1, a->r, b);
        pull(a);
    } else {
        b = t;
        split(t->l, k, a, b->l);
        pull(b);
    }
}

treap* merge(treap* a, treap* b) {
    if (a == NULL)
        return push(b);
    else if (b == NULL)
        return push(a);
    if (a->p > b->p) {
        push(a);
        a->r = merge(a->r, b);
        pull(a);
        return a;
    } else {
        push(b);
        b->l = merge(a, b->l);
        pull(b);
        return b;
    }
}

inline void slice(treap* t, int x, int y, treap*& l, treap*& m, treap*& r) {
    split(t, x - 1, l, r);
    split(r, y - x + 1, m, r);
}

treap* build(int n) {
    treap* r = NULL;
    int v;
    stack<treap*> rc;
    treap* nt;
    while (n--) {
        scanf("%d", &v);
        nt = New(v);
        r = NULL;
        while (!rc.empty() && rc.top()->p < nt->p) {
            pull(r = rc.top());
            rc.pop();
        }
        nt->l = r;
        if (!rc.empty())
            rc.top()->r = nt;
        rc.push(nt);
    }
    while (!rc.empty()) {
        pull(r = rc.top());
        rc.pop();
    }
    return r;
}

int main()
{
    srand(42);
    int n, q;
    char cmd[10];
    int x, y, v;
    treap *l, *m, *r;
    treap *ml, *mr;
    treap* root;

    while (scanf("%d", &n) == 1) {
        init();
        root = build(n);
        scanf("%d", &q);
        while (q--) {
            scanf("%s", cmd);
            switch (cmd[0]) {
            case 'A':
                scanf("%d%d%d", &x, &y, &v);
                slice(root, x, y, l, m, r);
                addn(m, v);
                root = merge(merge(l, m), r);
                break;
            case 'I':
                scanf("%d%d", &x, &v);
                split(root, x, l, r);
                root = merge(merge(l, New(v)), r);
                break;
            case 'D':
                scanf("%d", &x);
                slice(root, x, x, l, m, r);
                Del(m);
                root = merge(l, r);
                break;
            case 'M':
                scanf("%d%d", &x, &y);
                slice(root, x, y, l, m, r);
                printf("%d\n", m->mn);
                root = merge(merge(l, m), r);
                break;
            case 'R':
                scanf("%d%d", &x, &y);
                switch (cmd[3]) {
                case 'E':
                    slice(root, x, y, l, m, r);
                    reverse(m);
                    root = merge(merge(l, m), r);
                    break;
                case 'O':
                    scanf("%d", &v);
                    int len = (y-x+1);
                    v = (v % len + len) % len; // v could be negative?
                    if (v) {
                        slice(root, x, y, l, m, r);
                        split(m, len-v, ml, mr);
                        root = merge(merge(l, merge(mr, ml)), r);
                    }
                    break;
                }
                break;
            }
        }
    }
    return 0;
}
\end{minted}
\end{code}
\section{FFT}
\begin{code}
\begin{minted}
#include <bits/stdc++.h>

using namespace std;

typedef complex<double> cd;
const double PI = acos(-1);

void fft(vector<cd>& a, bool invert){
    int n = a.size();
    if(n == 1){
        return;
    }
    vector<cd> a0(n/2), a1(n/2);
    for(int i=0;2*i<n;i++){
        a0[i] = a[2*i];
        a1[i] = a[2*i+1];
    }
    fft(a0, invert);
    fft(a1, invert);

    double ang = 2 * PI / n * (invert ? -1 : 1);
    cd w(1), wn(cos(ang), sin(ang));
    for(int i=0;2*i<n;i++){
        a[i] = a0[i] + w  * a1[i];
        a[i+n/2] = a0[i] - w * a1[i];
        if(invert){
            a[i] /= 2;
            a[i+n/2] /= 2;
        }
        w *= wn;
    }
}

vector<int> multiply(vector<int>& a, vector<int>& b){
    vector<cd> fa(a.begin(), a.end()), fb(b.begin(), b.end());
    int n = 1;
    while(n < (int)a.size() + (int)b.size()){
        n <<= 1;
    }
    fa.resize(n);
    fb.resize(n);

    fft(fa, false);
    fft(fb, false);

    for(int i=0;i<n;i++){
        fa[i] *= fb[i];
    }

    fft(fa, true);

    vector<int> result(n);
    for(int i=0;i<n;i++){
        result[i] = round(fa[i].real());
    }

    return result;
}

int main(){
    vector<int> a{1, 2, 1};
    vector<int> b{2, 4, 6};

    auto res = multiply(a, b);
    for(auto i: res){
        cout << i << " " ;
    }
    cout << endl;
}
\end{minted}
\end{code}
\section{pbds ordered set}
\begin{code}
\begin{minted}
// C++ program to demonstrate the
// ordered set in GNU C++
#include <iostream>
using namespace std;

// Header files, namespaces,
// macros as defined above
#include <ext/pb_ds/assoc_container.hpp>
#include <ext/pb_ds/tree_policy.hpp>
using namespace __gnu_pbds;

#define ordered_set tree<int, null_type,less<int>, rb_tree_tag,tree_order_statistics_node_update>

// Driver program to test above functions
int main()
{
    // Ordered set declared with name o_set
    ordered_set o_set;

    // insert function to insert in
    // ordered set same as SET STL
    o_set.insert(5);
    o_set.insert(1);
    o_set.insert(2);

    // Finding the second smallest element
    // in the set using * because
    // find_by_order returns an iterator
    cout << *(o_set.find_by_order(1))
        << endl;

    // Finding the number of elements
    // strictly less than k=4
    cout << o_set.order_of_key(4)
        << endl;

    // Finding the count of elements less
    // than or equal to 4 i.e. strictly less
    // than 5 if integers are present
    cout << o_set.order_of_key(5)
        << endl;

    // Deleting 2 from the set if it exists
    if (o_set.find(2) != o_set.end())
        o_set.erase(o_set.find(2));

    // Now after deleting 2 from the set
    // Finding the second smallest element in the set
    cout << *(o_set.find_by_order(1))
        << endl;

    // Finding the number of
    // elements strictly less than k=4
    cout << o_set.order_of_key(4)
        << endl;

    return 0;
}
\end{minted}
\end{code}
\section{Almost union-find}
\begin{code}
\begin{minted}
vector<int> dsu(MAXN);
vector<int> convert(MAXN);
vector<ll> sz(MAXN, 1);
vector<ll> sum(MAXN, 0);

int n, q;

int find(int a){
    if(dsu[a] == a){
        return a;
    }
    return dsu[a] = find(dsu[a]);
}


void merge(int a, int b){
    int fa = find(a);
    int fb = find(b);
    if(fa == fb){
        return;
    } else {
        sum[fb] += sum[fa];
        sz[fb] += sz[fa];
        dsu[fa] = fb;
    }
}

void move(int a, int b){
    int fa = find(convert[a]);
    int fb = find(convert[b]);
    if(fa == fb){
        return;
    } else {
        sz[fa] -= 1;
        sum[fa] -= a;
        convert[a] = ++n;
        sum[convert[a]] = a;
        sz[convert[a]] = 1;
        merge(convert[a], convert[b]);
    }
}

void solve(){
    while(cin >> n >> q){
        for(int i=1;i<MAXN;i++){
            dsu[i] = sum[i] = convert[i] = i;
            sz[i] = 1;
        }
        int a, b, c;
        while(q--){
            cin >> c;
            if(c == 1){
                cin >> a >> b;
                merge(convert[a], convert[b]);
            } else if(c == 2){
                cin >> a >> b;
                move(a, b);
            } else {
                cin >> a;
                a = find(convert[a]);
                cout << sz[a] << " " << sum[a] << endl;
            }
        }
    }
    
}

/********** Good Luck :) **********/
int main () {
    TIME(main);
    IOS();
    int t = 1;
    // cin >> t;
    while(t--){
        solve();
    }

    return 0;
}	
\end{minted}
\end{code}
\section{Mega inversion}
\begin{code}
\begin{minted}
inline ll lowbit(ll x){
    return x & (-x);
}

int n;
vector<int> a;
class BIT{
private:
    vector<ll> sum;
public:
    BIT(): sum(vector<ll>(MAXN, 0)) {}
    void update(int idx, int val){
        for(int i=idx;i<=n;i+=lowbit(i)){
            sum[i] += val;
        }
    }
    ll query(int idx){
        ll res = 0;
        for(int i=idx;i>0;i-=lowbit(i)){
            res += sum[i];
        }

        return res;
    }
};

void solve(){
    cin >> n;
    a.resize(n);
    for(auto &i: a) cin >> i;
    BIT bit1, bit2;
    vector<ll> ans1(MAXN, 0);
    vector<ll> ans2(MAXN, 0);
    for(int i=n-1;i>=0;i--){
        ans1[i] = bit1.query(a[i]-1);
        bit1.update(a[i], 1);
    }
    for(int i=0;i<n;i++){
        ans2[i] = i - bit2.query(a[i]);
        bit2.update(a[i], 1);
    }
    ll ans = 0;
    for(int i=0;i<n;i++){
        ans += ans1[i] * ans2[i];
    }
    cout << ans << endl;

}

/********** Good Luck :) **********/
int main () {
    TIME(main);
    IOS();
    int t = 1;
    // cin >> t;
    while(t--){
        solve();
    }

    return 0;
}
\end{minted}
\end{code}
\section{逆序數對}
\begin{code}
\begin{minted}
#include <iostream>
#include <vector>
#include <algorithm>
using namespace std;
 
int Case, n, a[100005], bit[100005];
long long ans;
vector <int> v;
 
int get_id(int x){
    return lower_bound(v.begin(), v.end(), x)-v.begin()+1;
}
void update(int x){
    while (x <= n){
        bit[x]++;
        x += x & (-x);
    }
}
int query(int x){
    int ret = 0;
    while (x){
        ret += bit[x];
        x -= x & (-x);
    }
    return ret;
}
 
int main() {
    ios_base::sync_with_stdio(0);
    cin.tie(0);
    while (cin >> n){
        if (n == 0) break;
        v.clear();
        ans = 0;
        for (int i = 0; i < n; i++){
            cin >> a[i];
            bit[i] = 0;
            v.push_back(a[i]);
        }
        bit[n] = 0;
        sort(v.begin(), v.end());
        v.erase(unique(v.begin(), v.end()), v.end());
        for (int i = 0; i < n; i++){
            ans += i-query(get_id(a[i]));
            update(get_id(a[i]));
        }
        Case++;
        cout << "Case #" << Case << ": " << ans << "\n";
    }
}
\end{minted}
\end{code}
\section{Batmanacci}
\begin{code}
\begin{minted}
n, k = input().split()
n = int(n)
k = int(k)

fib = [0, 1, 1]

for i in range(3, n+1):
    fib.append(fib[i-1] + fib[i-2])

while n > 2:
    if k > fib[n-2]:
        k -= fib[n-2]
        n -= 1
    else:
        n -= 2
    
if n == 1:
    print("N")
else:
    print("A")
\end{minted}
\end{code}
\section{模逆元}
\begin{code}
\begin{minted}
#include <iostream>
using namespace std;
#define ll long long
 
const int mod = 17, maxn = 20;
 
ll pre[maxn+1];
ll inv[maxn+1];
ll prei[maxn+1];
 
void build(int n){
    pre[1] = pre[0] = 1, inv[1] = inv[0] = 1, prei[1] = prei[0] = 1;
    for(int i = 2 ; i <= n ; i++){
        pre[i] = pre[i-1] * i % mod;
        // i 的逆元inv[i]= -(p/i) * inv[p%i] (mod p)
        inv[i] = mod - (mod / i * inv[mod % i]) % mod;
        prei[i] = prei[i-1] * inv[i] % mod;
    }
}
 
ll C(int n, int k){
   return pre[n] * prei[k] % mod * prei[n-k] % mod;
}
 
int main(){
    build(maxn);
    cout << inv[10] << endl;
    cout << C(6, 3) << endl;
}
\end{minted}
\end{code}
\section{樹重心}
\begin{code}
\begin{minted}
#include <iostream>
#include <vector>
using namespace std;
const int maxn = 100005;
int N;
vector <int> g[maxn];
int cost[maxn]; //慘度
 
int dfs(int now, int pre){
    //tot: 以now為root的子樹size
    int tot = 1, ret = 0;
    // now下方的子樹
    for (auto nxt: g[now]){
        if (nxt != pre){
            ret = dfs(nxt, now);
            tot += ret;
            cost[now] = max(cost[now], ret);
        }
    }
    // now頭上的子樹
    cost[now] = max(cost[now], N - tot);
    return tot;
}
 
int main() {
    ios_base::sync_with_stdio(0);
    cin.tie(0);
    int T, a, b;
    cin >> T;
    while (T--){
        cin >> N;
        for (int i = 0; i < N; i++){
            cost[i] = 0;
            g[i].clear();
        }
        for (int i = 0; i < N-1; i++){
            cin >> a >> b;
            g[a].push_back(b);
            g[b].push_back(a);
        }
        dfs(0, -1); //將0當作root
        int mn = 0x7FFFFFFF;
        int ans = -1;
        for (int i = 0; i < N; i++){
            if (cost[i] < mn){
                mn = cost[i];
                ans = i;
            }
        }
        cout << ans << "\n";
    }
    return 0;
}
\end{minted}
\end{code}
\begin{code}
\begin{minted}

// 这份代码默认节点编号从 1 开始,即 i ∈ [1,n]
int size[MAXN],  // 这个节点的“大小”(所有子树上节点数 + 该节点)
    weight[MAXN],  // 这个节点的“重量”
    centroid[2];   // 用于记录树的重心(存的是节点编号)

void GetCentroid(int cur, int fa) {  // cur 表示当前节点 (current)
  size[cur] = 1;
  weight[cur] = 0;
  for (int i = head[cur]; i != -1; i = e[i].nxt) {
    if (e[i].to != fa) {  // e[i].to 表示这条有向边所通向的节点。
      GetCentroid(e[i].to, cur);
      size[cur] += size[e[i].to];
      weight[cur] = max(weight[cur], size[e[i].to]);
    }
  }
  weight[cur] = max(weight[cur], n - size[cur]);
  if (weight[cur] <= n / 2) {  // 依照树的重心的定义统计
    centroid[centroid[0] != 0] = cur;
  }
}
\end{minted}
\end{code}
\section{pbds hashtable}
\begin{code}
\begin{minted}
#include <ext/pb_ds/assoc_container.hpp>
using namespace __gnu_pbds;
gp_hash_table<int, int> table;
cc_hash_table<int, int> table;
\end{minted}
\end{code}

\section{0/1 背包}
\begin{code}
\begin{minted}
// :80 <enter>
#include <bits/stdc++.h>
using namespace std;
typedef long long ll;
typedef pair<int, int> pii;
typedef pair<ll, ll> pll;
typedef pair<int, ll> pil;
typedef pair<ll, int> pli;
typedef pair<double,double> pdd;
#define SQ(i) ((i)*(i))
#define MEM(a, b) memset(a, (b), sizeof(a))
#define SZ(i) int(i.size())
#define FOR(i, j, k, in) for (int i=j ; i<(k) ; i+=in)
#define RFOR(i, j, k, in) for (int i=j ; i>=(k) ; i-=in)
#define REP(i, j) FOR(i, 0, j, 1)
#define REP1(i,j) FOR(i, 1, j+1, 1)
#define RREP(i, j) RFOR(i, j, 0, 1)
#define ALL(_a) _a.begin(),_a.end()
#define mp make_pair
#define pb push_back
#define eb emplace_back
#define X first
#define Y second
#ifdef jayinnn
#define TIME(i) Timer i(#i)
#define debug(...) do{\
    fprintf(stderr,"%s - %d (%s) = ",__PRETTY_FUNCTION__,__LINE__,#__VA_ARGS__);\
    _do(__VA_ARGS__);\
}while(0)
template<typename T>void _do(T &&_x){cerr<<_x<<endl;}
template<typename T,typename ...S> void _do(T &&_x,S &&..._t){cerr<<_x<<", ";_do(_t...);}
template<typename _a,typename _b> ostream& operator << (ostream &_s,const pair<_a,_b> &_p){return _s<<"("<<_p.X<<","<<_p.Y<<")";}
template<typename It> ostream& _OUTC(ostream &_s,It _ita,It _itb)
{
    _s<<"{";
    for(It _it=_ita;_it!=_itb;_it++)
    {
        _s<<(_it==_ita?"":",")<<*_it;
    }
    _s<<"}";
    return _s;
}
template<typename _a> ostream &operator << (ostream &_s,vector<_a> &_c){return _OUTC(_s,ALL(_c));}
template<typename _a> ostream &operator << (ostream &_s,set<_a> &_c){return _OUTC(_s,ALL(_c));}
template<typename _a> ostream &operator << (ostream &_s,deque<_a> &_c){return _OUTC(_s,ALL(_c));}
template<typename _a,typename _b> ostream &operator << (ostream &_s,map<_a,_b> &_c){return _OUTC(_s,ALL(_c));}
template<typename _t> void pary(_t _a,_t _b){_OUTC(cerr,_a,_b);cerr<<endl;}
#define IOS()
class Timer {
private:
    string scope_name;
    chrono::high_resolution_clock::time_point start_time;
public:
    Timer (string name) : scope_name(name) {
        start_time = chrono::high_resolution_clock::now();
    }
    ~Timer () {
        auto stop_time = chrono::high_resolution_clock::now();
        auto length = chrono::duration_cast<chrono::microseconds>(stop_time - start_time).count();
        double mlength = double(length) * 0.001;
        debug(scope_name, mlength);
    }
};
#else
#define TIME(i)
#define debug(...)
#define pary(...)
#define endl '\n'
#define IOS() ios_base::sync_with_stdio(0);cin.tie(0)
#endif

const ll MOD = 1000000007;
const ll INF = 0x3f3f3f3f3f3f3f3f;
const int iNF = 0x3f3f3f3f;
const ll MAXN = 100005;

mt19937 rng(chrono::steady_clock::now().time_since_epoch().count());

void solve(int c, int n){
    vector<int> v(n);
    vector<int> w(n);
    REP(i, n){
        cin >> v[i] >> w[i];
    }
    vector<vector<int>> dp(n+1, vector<int>(c+1, 0));
    vector<vector<bool>> has(n+1, vector<bool>(c+1, false));
    for(int i=0;i<n;i++){
        for(int j=0;j<=c;j++){
            if(j - w[i] < 0){
                dp[i+1][j] = dp[i][j];
            } else {
                if(dp[i][j-w[i]] + v[i] > dp[i][j]){
                    dp[i+1][j] = dp[i][j-w[i]] + v[i];
                    has[i][j] = true;
                } else {
                    dp[i+1][j] = dp[i][j];
                }
            }
        }
    }
    vector<int> ans;
    for(int i=n-1, j=c;i>=0;i--){
        if(has[i][j]){
            ans.push_back(i);
            j -= w[i];
        }
    }
    cout << ans.size() << endl;
    for(auto i: ans) cout << i << " ";
    cout << endl;
}

/********** Good Luck :) **********/
int main () {
    TIME(main);
    IOS();
    // cin >> t;
    int c, n;
    while(cin >> c >> n){
        solve(c, n);
    }

    return 0;
}
\end{minted}
\end{code}
\section{有限背包}
\begin{code}
\begin{minted}
const int N = 100, W = 100000;
int cost[N], weight[N], number[N];
int c[W + 1];

void knapsack(int n, int w)
{
    for (int i = 0; i < n; ++i)
    {
        int num = min(number[i], w / weight[i]);
        for (int k = 1; num > 0; k *= 2)
        {
            if (k > num) k = num;
            num -= k;
            for (int j = w; j >= weight[i] * k; --j)
                c[j] = max(c[j], c[j - weight[i] * k] + cost[i] * k);
        }
    }
    cout << "最高的價值為" << c[w];
}
\end{minted}
\end{code}
\section{無限背包}
$$ c(n, w) = max( c(n-1, w), c(n, w-weight[n]) + cost[n] )$$
\begin{code}
\begin{minted}
const int N = 100, W = 100000;
int cost[N], weight[N];
int c[W + 1];

void knapsack(int n, int w)
{
    memset(c, 0, sizeof(c));

    for (int i=0; i<n; ++i)
        for (int j = weight[i]; j <= w; ++j)
            c[j] = max(c[j], c[j - weight[i]] + cost[i]);

    cout << "最高的價值為" << c[w];
}
\end{minted}
\end{code}
\section{換錢問題}
\begin{code}
\begin{minted}
// :80 <enter>
#include <bits/stdc++.h>
using namespace std;
typedef long long ll;
typedef pair<int, int> pii;
typedef pair<ll, ll> pll;
typedef pair<int, ll> pil;
typedef pair<ll, int> pli;
typedef pair<double,double> pdd;
#define SQ(i) ((i)*(i))
#define MEM(a, b) memset(a, (b), sizeof(a))
#define SZ(i) int(i.size())
#define FOR(i, j, k, in) for (int i=j ; i<(k) ; i+=in)
#define RFOR(i, j, k, in) for (int i=j ; i>=(k) ; i-=in)
#define REP(i, j) FOR(i, 0, j, 1)
#define REP1(i,j) FOR(i, 1, j+1, 1)
#define RREP(i, j) RFOR(i, j, 0, 1)
#define ALL(_a) _a.begin(),_a.end()
#define mp make_pair
#define pb push_back
#define eb emplace_back
#define X first
#define Y second
#ifdef jayinnn
#define TIME(i) Timer i(#i)
#define debug(...) do{\
    fprintf(stderr,"%s - %d (%s) = ",__PRETTY_FUNCTION__,__LINE__,#__VA_ARGS__);\
    _do(__VA_ARGS__);\
}while(0)
template<typename T>void _do(T &&_x){cerr<<_x<<endl;}
template<typename T,typename ...S> void _do(T &&_x,S &&..._t){cerr<<_x<<", ";_do(_t...);}
template<typename _a,typename _b> ostream& operator << (ostream &_s,const pair<_a,_b> &_p){return _s<<"("<<_p.X<<","<<_p.Y<<")";}
template<typename It> ostream& _OUTC(ostream &_s,It _ita,It _itb)
{
    _s<<"{";
    for(It _it=_ita;_it!=_itb;_it++)
    {
        _s<<(_it==_ita?"":",")<<*_it;
    }
    _s<<"}";
    return _s;
}
template<typename _a> ostream &operator << (ostream &_s,vector<_a> &_c){return _OUTC(_s,ALL(_c));}
template<typename _a> ostream &operator << (ostream &_s,set<_a> &_c){return _OUTC(_s,ALL(_c));}
template<typename _a> ostream &operator << (ostream &_s,deque<_a> &_c){return _OUTC(_s,ALL(_c));}
template<typename _a,typename _b> ostream &operator << (ostream &_s,map<_a,_b> &_c){return _OUTC(_s,ALL(_c));}
template<typename _t> void pary(_t _a,_t _b){_OUTC(cerr,_a,_b);cerr<<endl;}
#define IOS()
class Timer {
private:
    string scope_name;
    chrono::high_resolution_clock::time_point start_time;
public:
    Timer (string name) : scope_name(name) {
        start_time = chrono::high_resolution_clock::now();
    }
    ~Timer () {
        auto stop_time = chrono::high_resolution_clock::now();
        auto length = chrono::duration_cast<chrono::microseconds>(stop_time - start_time).count();
        double mlength = double(length) * 0.001;
        debug(scope_name, mlength);
    }
};
#else
#define TIME(i)
#define debug(...)
#define pary(...)
#define endl '\n'
#define IOS() ios_base::sync_with_stdio(0);cin.tie(0)
#endif

const ll MOD = 1000000007;
const ll INF = 0x3f3f3f3f3f3f3f3f;
const int iNF = 0x3f3f3f3f;
const ll MAXN = 20005;

mt19937 rng(chrono::steady_clock::now().time_since_epoch().count());


void solve(){
    vector<int> dp(MAXN, iNF);
    int x;
    cin >> x;
    int n;
    cin >> n;
    vector<int> a(n);
    REP(i, n) cin >> a[i];
    dp[0] = 0;
    for(int i=0;i<n;i++){
        for(int j=MAXN;j>=0;j--){
            if(j - a[i] >= 0){
                dp[j] = min(dp[j], dp[j-a[i]] + 1);
            }
        }
    }
    for(int i=x;i<MAXN;i++){
        if(dp[i] != iNF){
            cout << i << " " << dp[i] << endl;
            return;
        }
    }
}

/********** Good Luck :) **********/
int main () {
    TIME(main);
    IOS();
    int t = 1;
    cin >> t;
    while(t--){
        solve();
    }

    return 0;
}
\end{minted}
\end{code}
\section{Increasing Subsequences}
\begin{code}
\begin{minted}
// :80 <enter>
#include <bits/stdc++.h>
using namespace std;
typedef long long ll;
typedef pair<int, int> pii;
typedef pair<ll, ll> pll;
typedef pair<int, ll> pil;
typedef pair<ll, int> pli;
typedef pair<double,double> pdd;
#define SQ(i) ((i)*(i))
#define MEM(a, b) memset(a, (b), sizeof(a))
#define SZ(i) int(i.size())
#define FOR(i, j, k, in) for (int i=j ; i<(k) ; i+=in)
#define RFOR(i, j, k, in) for (int i=j ; i>=(k) ; i-=in)
#define REP(i, j) FOR(i, 0, j, 1)
#define REP1(i,j) FOR(i, 1, j+1, 1)
#define RREP(i, j) RFOR(i, j, 0, 1)
#define ALL(_a) _a.begin(),_a.end()
#define mp make_pair
#define pb push_back
#define eb emplace_back
#define X first
#define Y second
#ifdef jayinnn
#define TIME(i) Timer i(#i)
#define debug(...) do{\
    fprintf(stderr,"%s - %d (%s) = ",__PRETTY_FUNCTION__,__LINE__,#__VA_ARGS__);\
    _do(__VA_ARGS__);\
}while(0)
template<typename T>void _do(T &&_x){cerr<<_x<<endl;}
template<typename T,typename ...S> void _do(T &&_x,S &&..._t){cerr<<_x<<", ";_do(_t...);}
template<typename _a,typename _b> ostream& operator << (ostream &_s,const pair<_a,_b> &_p){return _s<<"("<<_p.X<<","<<_p.Y<<")";}
template<typename It> ostream& _OUTC(ostream &_s,It _ita,It _itb)
{
    _s<<"{";
    for(It _it=_ita;_it!=_itb;_it++)
    {
        _s<<(_it==_ita?"":",")<<*_it;
    }
    _s<<"}";
    return _s;
}
template<typename _a> ostream &operator << (ostream &_s,vector<_a> &_c){return _OUTC(_s,ALL(_c));}
template<typename _a> ostream &operator << (ostream &_s,set<_a> &_c){return _OUTC(_s,ALL(_c));}
template<typename _a> ostream &operator << (ostream &_s,deque<_a> &_c){return _OUTC(_s,ALL(_c));}
template<typename _a,typename _b> ostream &operator << (ostream &_s,map<_a,_b> &_c){return _OUTC(_s,ALL(_c));}
template<typename _t> void pary(_t _a,_t _b){_OUTC(cerr,_a,_b);cerr<<endl;}
#define IOS()
class Timer {
private:
    string scope_name;
    chrono::high_resolution_clock::time_point start_time;
public:
    Timer (string name) : scope_name(name) {
        start_time = chrono::high_resolution_clock::now();
    }
    ~Timer () {
        auto stop_time = chrono::high_resolution_clock::now();
        auto length = chrono::duration_cast<chrono::microseconds>(stop_time - start_time).count();
        double mlength = double(length) * 0.001;
        debug(scope_name, mlength);
    }
};
#else
#define TIME(i)
#define debug(...)
#define pary(...)
#define endl '\n'
#define IOS() ios_base::sync_with_stdio(0);cin.tie(0)
#endif

const ll MOD = 1000000007;
const ll INF = 0x3f3f3f3f3f3f3f3f;
const int iNF = 0x3f3f3f3f;
const ll MAXN = 100005;

mt19937 rng(chrono::steady_clock::now().time_since_epoch().count());

void print(vector<int>& a, vector<int>& prev, int idx){
    vector<int> tmp;
    while(idx != -1){
        tmp.push_back(a[idx]);
        idx = prev[idx];
    }
    reverse(ALL(tmp));
    for(auto i: tmp) cout << i << " ";
    cout << endl;
}

void solve(){
    int n;
    cin >> n;
    if(n == 0) exit(0);
    vector<int> a(n);
    for(auto &i: a) cin >> i;
    vector<int> dp(n, 1);
    vector<int> prev(n, -1);
    for(int i=0;i<n;i++){
        for(int j=0;j<i;j++){
            if(a[j] < a[i] and (dp[j] + 1 > dp[i] or (dp[j] + 1 == dp[i] and (prev[i] != -1 and a[prev[i]] > a[j])))){
                dp[i] = dp[j] + 1;
                prev[i] = j;
            }
        }
    }
    int ans = 0;
    int ans_val = iNF;
    for(int i=0;i<n;i++){
        if(ans < dp[i] or (ans == dp[i] and ans_val > a[i])){
            ans = dp[i];
            ans_val = a[i];
        }
    }
    cout << ans << " ";
    for(int i=0;i<n;i++){
        if(dp[i] == ans and ans_val == a[i]){
            print(a, prev, i);
            break;
        }
    }
}

/********** Good Luck :) **********/
int main () {
    TIME(main);
    IOS();
    // cin >> t;
    while(true){
        solve();
    }

    return 0;
}
\end{minted}
\end{code}
\section{Exgcd}
\begin{code}
\begin{minted}
#include <bits/stdc++.h>
using namespace std;
typedef long long ll;
typedef pair<int, int> pii;
typedef pair<ll, ll> pll;
typedef pair<int, ll> pil;
typedef pair<ll, int> pli;
typedef pair<double,double> pdd;
#define SQ(i) ((i)*(i))
#define MEM(a, b) memset(a, (b), sizeof(a))
#define SZ(i) int(i.size())
#define FOR(i, j, k, in) for (int i=j ; i<(k) ; i+=in)
#define RFOR(i, j, k, in) for (int i=j ; i>=(k) ; i-=in)
#define REP(i, j) FOR(i, 0, j, 1)
#define REP1(i,j) FOR(i, 1, j+1, 1)
#define RREP(i, j) RFOR(i, j, 0, 1)
#define ALL(_a) _a.begin(),_a.end()
#define mp make_pair
#define pb push_back
#define eb emplace_back
#define X first
#define Y second
#ifdef tmd
#define TIME(i) Timer i(#i)
#define debug(...) do{\
    fprintf(stderr,"%s - %d (%s) = ",__PRETTY_FUNCTION__,__LINE__,#__VA_ARGS__);\
    _do(__VA_ARGS__);\
}while(0)
template<typename T>void _do(T &&_x){cerr<<_x<<endl;}
template<typename T,typename ...S> void _do(T &&_x,S &&..._t){cerr<<_x<<", ";_do(_t...);}
template<typename _a,typename _b> ostream& operator << (ostream &_s,const pair<_a,_b> &_p){return _s<<"("<<_p.X<<","<<_p.Y<<")";}
template<typename It> ostream& _OUTC(ostream &_s,It _ita,It _itb)
{
    _s<<"{";
    for(It _it=_ita;_it!=_itb;_it++)
    {
        _s<<(_it==_ita?"":",")<<*_it;
    }
    _s<<"}";
    return _s;
}
template<typename _a> ostream &operator << (ostream &_s,vector<_a> &_c){return _OUTC(_s,ALL(_c));}
template<typename _a> ostream &operator << (ostream &_s,set<_a> &_c){return _OUTC(_s,ALL(_c));}
template<typename _a> ostream &operator << (ostream &_s,deque<_a> &_c){return _OUTC(_s,ALL(_c));}
template<typename _a,typename _b> ostream &operator << (ostream &_s,map<_a,_b> &_c){return _OUTC(_s,ALL(_c));}
template<typename _t> void pary(_t _a,_t _b){_OUTC(cerr,_a,_b);cerr<<endl;}
#define IOS()
class Timer {
private:
    string scope_name;
    chrono::high_resolution_clock::time_point start_time;
public:
    Timer (string name) : scope_name(name) {
        start_time = chrono::high_resolution_clock::now();
    }
    ~Timer () {
        auto stop_time = chrono::high_resolution_clock::now();
        auto length = chrono::duration_cast<chrono::microseconds>(stop_time - start_time).count();
        double mlength = double(length) * 0.001;
        debug(scope_name, mlength);
    }
};
#else
#define TIME(i)
#define debug(...)
#define pary(...)
#define endl '\n'
#define IOS() ios_base::sync_with_stdio(0);cin.tie(0)
#endif

const ll MOD = 1000000007;
const ll INF = 0x3f3f3f3f3f3f3f3f;
const int iNF = 0x3f3f3f3f;
const ll MAXN = 100005;

mt19937 rng(chrono::steady_clock::now().time_since_epoch().count());


/* 
find solution of ax + by = gcd(a, b)

gcd(a, b) = gcd(b, a % b), therefore

bx + (a % b)y = gcd(b, a % b) = gcd(a, b) = ax + by
bx' + (a % b)y' = bx' + (a - floor(a/b) * b)y' = ay' + b(x' - floor(a/b)y') = ax + by

x = y', y = x' - floor(a/b) * y'
*/

void exgcd(int a, int b, int &x, int &y){
    if(b == 0){
        x = 1;y = 0;
        return;
    }
    int x1, y1;
    exgcd(b, a%b, x1, y1);
    x=y1;
    y=x1-(a/b)*y1;

}

// if x < 0, ans = (x + b) % b

/********** Good Luck :) **********/
int main () {
    TIME(main);
    IOS();

    return 0;
}
\end{minted}
\end{code}
\section{計算幾何模板}
\begin{code}
\begin{minted}
#include <bits/stdc++.h>
using namespace std;
#define X first
#define Y second
typedef pair<double, double> Pt;


Pt operator+(const Pt& p1, const Pt& p2){
    return Pt(p1.X + p2.X, p1.Y + p2.Y);
}
Pt operator­( const Pt& p1 , const Pt& p2 ){
    return Pt( p1.X - p2.X , p1.Y - p2.Y );
}
double operator*( const Pt& p1 , const Pt& p2 ){
    return p1.X * p2.X + p1.Y * p2.Y;
}

double operator^( const Pt& p1 , const Pt& p2 ){
    return p1.X * p2.Y - p1.Y * p2.X;
}

int main(){

}
\end{minted}
\end{code}

\section{Monotone Chain}
\begin{code}
\begin{minted}
// :80 <enter>
#include <bits/stdc++.h>
using namespace std;
typedef long long ll;
typedef pair<int, int> pii;
typedef pair<ll, ll> pll;
typedef pair<int, ll> pil;
typedef pair<ll, int> pli;
typedef pair<double,double> pdd;
#define SQ(i) ((i)*(i))
#define MEM(a, b) memset(a, (b), sizeof(a))
#define SZ(i) int(i.size())
#define FOR(i, j, k, in) for (int i=j ; i<(k) ; i+=in)
#define RFOR(i, j, k, in) for (int i=j ; i>=(k) ; i-=in)
#define REP(i, j) FOR(i, 0, j, 1)
#define REP1(i,j) FOR(i, 1, j+1, 1)
#define RREP(i, j) RFOR(i, j, 0, 1)
#define ALL(_a) _a.begin(),_a.end()
#define mp make_pair
#define pb push_back
#define eb emplace_back
#define X first
#define Y second
#ifdef jayinnn
#define TIME(i) Timer i(#i)
#define debug(...) do{\
    fprintf(stderr,"%s - %d (%s) = ",__PRETTY_FUNCTION__,__LINE__,#__VA_ARGS__);\
    _do(__VA_ARGS__);\
}while(0)
template<typename T>void _do(T &&_x){cerr<<_x<<endl;}
template<typename T,typename ...S> void _do(T &&_x,S &&..._t){cerr<<_x<<", ";_do(_t...);}
template<typename _a,typename _b> ostream& operator << (ostream &_s,const pair<_a,_b> &_p){return _s<<"("<<_p.X<<","<<_p.Y<<")";}
template<typename It> ostream& _OUTC(ostream &_s,It _ita,It _itb)
{
    _s<<"{";
    for(It _it=_ita;_it!=_itb;_it++)
    {
        _s<<(_it==_ita?"":",")<<*_it;
    }
    _s<<"}";
    return _s;
}
template<typename _a> ostream &operator << (ostream &_s,vector<_a> &_c){return _OUTC(_s,ALL(_c));}
template<typename _a> ostream &operator << (ostream &_s,set<_a> &_c){return _OUTC(_s,ALL(_c));}
template<typename _a> ostream &operator << (ostream &_s,deque<_a> &_c){return _OUTC(_s,ALL(_c));}
template<typename _a,typename _b> ostream &operator << (ostream &_s,map<_a,_b> &_c){return _OUTC(_s,ALL(_c));}
template<typename _t> void pary(_t _a,_t _b){_OUTC(cerr,_a,_b);cerr<<endl;}
#define IOS()
class Timer {
private:
    string scope_name;
    chrono::high_resolution_clock::time_point start_time;
public:
    Timer (string name) : scope_name(name) {
        start_time = chrono::high_resolution_clock::now();
    }
    ~Timer () {
        auto stop_time = chrono::high_resolution_clock::now();
        auto length = chrono::duration_cast<chrono::microseconds>(stop_time - start_time).count();
        double mlength = double(length) * 0.001;
        debug(scope_name, mlength);
    }
};
#else
#define TIME(i)
#define debug(...)
#define pary(...)
#define endl '\n'
#define IOS() ios_base::sync_with_stdio(0);cin.tie(0)
#endif

const ll MOD = 1000000007;
const ll INF = 0x3f3f3f3f3f3f3f3f;
const int iNF = 0x3f3f3f3f;
const ll MAXN = 100005;

mt19937 rng(chrono::steady_clock::now().time_since_epoch().count());

int cross(pii o, pii a, pii b){
    int u = a.X - o.X;
    int v = a.Y - o.Y;
    int s = b.X - o.X;
    int t = b.Y - o.Y;

    return u * t - s * v;
}

bool cmp(pii &a, pii &b){
    return (a.X < b.X) || (a.X == b.X && a.Y < b.Y);
}

void solve(int n){
    vector<pii> a;
    set<pii> s;
    int x, y;
    for(int i=0;i<n;i++){
        cin >> x >> y;
        s.insert({x, y});
    }
    for(auto i: s){
        a.push_back(i);
    }
    n = a.size();
    if(n <= 2){
        cout << n << endl;
        for(auto i: a){
            cout << i.X << " " << i.Y << endl;
        }
        return;
    }
    sort(ALL(a), cmp);
    debug(a);
    debug(a);
    int idx = 0;
    vector<pii> ch(2*n);
    for(int i=0;i<n;i++){
        while(idx >= 2 && cross(ch[idx-2], ch[idx-1], a[i]) <= 0){
            idx --;
        }
        ch[idx++] = a[i];

    }
    for(int i=n-2,t=idx+1;i>=0;i--){
        while(idx >= t && cross(ch[idx-2], ch[idx-1], a[i]) <= 0){
            idx --;
        }
        ch[idx++] = a[i];
    }
    idx --;
    cout << idx << endl;
    for(int i=0;i<idx;i++){
        cout << ch[i].X << " " << ch[i].Y << endl;
    }

}



/********** Good Luck :) **********/
int main () {
    TIME(main);
    IOS();
    int n;
    
    while(cin >> n){
        if(n == 0) break;
        solve(n);
    }

    return 0;
}

\end{minted}
\end{code}
\section{旋轉卡尺}
\begin{code}
\begin{minted}
#include <iostream>
#include <cmath>
#include <algorithm>
#include <vector>
#include <map>
#include <stack>
#include <set>
#include <queue>
#include <list>
#include <string.h>
#include <complex>
#include <sstream>
using namespace std;
#define INITIO()    ios_base::sync_with_stdio(false);cin.tie(NULL);
#define FILE()      freopen("a.in","r",stdin);freopen("out.txt","w",stdout);
#define endl        '\n'
#define F           first
#define S           second
#define pb          push_back
#define pf          push_front
#define all(a)      a.begin(),a.end()
#define rall(a)     a.rbegin(),a.rend()
#define sz(a)       (int)a.size()
#define FOR(i,a,b)  for(int i = (a); i < (b); i++)
using vii = vector<vector<int>>;
using vi = vector<int>;
using mii = map<int,int>;
using cd = complex<double>;
typedef long long ll;
typedef unsigned long long ull;
typedef pair<int,int> pii;
typedef pair<long,long> pll;
const double PI   = acos(-1);
const ll Mod     = 1e9+7;
//const ll  Inf     = 2e18+9;
//const int N       = 1e5+9;
//const int dx[4] = {0, 1,  0, -1};
//const int dy[4] = {1, 0, -1,  0};
//const int dx8[8] = {-1, -1, -1, 0, 1, 1, 1, 0};
//const int dy8[8] = {-1, 0, 1, 1, 1, 0, -1, -1};
ll gcd(ll a, ll b){if( b==0 )return a;return gcd( b, a%b );}
ll lcm(ll a,ll b){return a * b / gcd(a, b);}
int mul(int a, int b) {return (1LL * a * b) % Mod;}
ll add(ll a, ll b) {a += b;if (a >= Mod) a -= Mod;if (a < 0) a += Mod;return a;}
ll sub(ll a, ll b) {return (a - b) % Mod + ((a >= b) ? 0 : Mod);}
// ----------------------------------------------------------------------------------------------------

ll cross(pll a,pll b,pll c){
    return (a.F-b.F)*(a.S-c.S) - (a.S-b.S)*(a.F-c.F);
}

void solve(){
    int n;
    scanf("%d", &n);
    vector<pll> a;
    set<pii> s;
    FOR(i,0,n){
        int x,y;
        scanf("%d %d", &x, &y);
        if(s.count({x,y}))continue;
        a.pb({x,y});
        s.insert({x,y});
    }
    n=sz(a);
    sort(all(a));
    vector<pll> ans;
    FOR(i,0,n){
        while(sz(ans)>=2 && cross(ans[sz(ans)-1], ans[sz(ans)-2], a[i]) <= 0)ans.pop_back();
        ans.pb(a[i]);
    }
    for(int i=n-2, t=sz(ans)+1;i>=0;i--){
        while(sz(ans) >= t && cross(ans[sz(ans)-1], ans[sz(ans)-2], a[i]) <= 0)ans.pop_back();
        ans.pb(a[i]);
    }
    if(sz(ans)>1)ans.pop_back();
    a = ans;
    n=sz(a);
    ll best=0;
    //FOR(i,0,n)cout<<a[i].F<<' '<<a[i].S<<endl;
    for(int i=0;i<n-2;i++){
        int k=i+1;
        for(int j=i+2;j<n;j++){
            ll area = abs(cross(a[i],a[k],a[j]));
            while(1){
                k++;
                ll newarea = abs(cross(a[i],a[k],a[j]));
                if(newarea <= area || k >= j)break;
                area = newarea;
            }
            k = max(i+1, k-1);
            best = max(area, best);
        }
    }
    printf("%.5f\n",0.5*best);
}

int main(){
    INITIO()
    //int t;cin>>t;
    //while(t--){
        solve();
    //}
    return 0;
}
\end{minted}
\end{code}

\section{Optimal Triangulation}
d(i, j) = max(d(i, k) + d(k, j) + w(i,j,k)|i<k<j) ,w(i,j,k) is the weight function
\begin{code}
\begin{minted}
int n;
vector<pii> a;
double area(int i,int j,int k){
    return abs((a[j].F - a[i].F)*(a[k].S - a[i].S)*1.0 - (a[j].S - a[i].S)*(a[k].F - a[i].F)*1.0)/2.0;
}

bool judge(int a,int b,int c){
    for(int i=0;i<n;i++){
        if(i==a || i==b || i==c)continue;
        double s = area(a,b,i) + area(a,c,i) + area(b,c,i) - area(a,b,c);
        if(fabs(s) < 0.01)return false;
    }
    return true;
}

double d[55][55];

double dp(int i,int j){
    if(i+1 >= j)return d[i][j]=0;
    if(d[i][j] != -1.0)return d[i][j];
    double ans=1e9*1.0;
    for(int k=i+1;k<j;k++){
        if(judge(i,j,k))ans = min(ans, max(dp(i,k),max(dp(k,j), area(i,k,j))));
    }
    return d[i][j] = ans;
}

int main(){
    ios_base::sync_with_stdio(false);cin.tie(NULL);
    //freopen("a.in","r",stdin);freopen("out.txt","w",stdout);
    int t;cin>>t;
    while(t--){
        cin>>n;
        a.resize(n);FOR(i,0,n)cin>>a[i].F>>a[i].S;
        FOR(i,0,n)FOR(j,0,n)d[i][j] = -1.0;
        printf("%.1f\n", dp(0,n-1));
        
    }
    return 0;
}
\end{minted}
\end{code}
\section{Catalan Number}
When counting the way of triangulation of a convex hull.
f(n) = f(n-1)*f(2) + f(3)*f(n-2) + … + f(n-1)*f(2)

\section{Optimal Matrix Chain Multiplication}
\begin{code}
\begin{minted}
#include <bits/stdc++.h>
using namespace std;

int p[105];

int dp(int i, int j){
    if(j <= i+1) return 0;
    int mn = 10000000;
    for(int k=i+1;k<j;k++){
        int cost = dp(i,k) + dp(k,j) + p[i]*p[k]*p[j];
        mn = min(cost, mn);
    }
    return mn;
}

int main(){
    int n;
    cin>>n;
    for(int i=0;i<n;i++)cin>>p[i];
    cout<<dp(0,n-1)<<endl;
    return 0;
}
\end{minted}
\end{code}

\section{Cutting Stick}
\begin{code}
\begin{minted}

\end{minted}
\end{code}

\section{Check if point belongs to the convex polygon (logN for each query)}
\begin{code}
\begin{minted}
struct pt {
    long long x, y;
    pt() {}
    pt(long long _x, long long _y) : x(_x), y(_y) {}
    pt operator+(const pt &p) const { return pt(x + p.x, y + p.y); }
    pt operator-(const pt &p) const { return pt(x - p.x, y - p.y); }
    long long cross(const pt &p) const { return x * p.y - y * p.x; }
    long long dot(const pt &p) const { return x * p.x + y * p.y; }
    long long cross(const pt &a, const pt &b) const { return (a - *this).cross(b - *this); }
    long long dot(const pt &a, const pt &b) const { return (a - *this).dot(b - *this); }
    long long sqrLen() const { return this->dot(*this); }
};

bool lexComp(const pt &l, const pt &r) {
    return l.x < r.x || (l.x == r.x && l.y < r.y);
}

int sgn(long long val) { return val > 0 ? 1 : (val == 0 ? 0 : -1); }

vector<pt> seq;
pt translation;
int n;

bool pointInTriangle(pt a, pt b, pt c, pt point) {
    long long s1 = abs(a.cross(b, c));
    long long s2 = abs(point.cross(a, b)) + abs(point.cross(b, c)) + abs(point.cross(c, a));
    return s1 == s2;
}

void prepare(vector<pt> &points) {
    n = points.size();
    int pos = 0;
    for (int i = 1; i < n; i++) {
        if (lexComp(points[i], points[pos]))
            pos = i;
    }
    rotate(points.begin(), points.begin() + pos, points.end());

    n--;
    seq.resize(n);
    for (int i = 0; i < n; i++)
        seq[i] = points[i + 1] - points[0];
    translation = points[0];
}

bool pointInConvexPolygon(pt point) {
    point = point - translation;
    if (seq[0].cross(point) != 1 &&
            sgn(seq[0].cross(point)) != sgn(seq[0].cross(seq[n - 1])))
        return false;
    if (seq[n - 1].cross(point) != 0 &&
            sgn(seq[n - 1].cross(point)) != sgn(seq[n - 1].cross(seq[0])))
        return false;

    if (seq[0].cross(point) == 0)
        return seq[0].sqrLen() >= point.sqrLen();

    int l = 0, r = n - 1;
    while (r - l > 1) {
        int mid = (l + r) / 2;
        int pos = mid;
        if (seq[pos].cross(point) >= 0)
            l = mid;
        else
            r = mid;
    }
    int pos = l;
    return pointInTriangle(seq[pos], seq[pos + 1], pt(0, 0), point);
}
\end{minted}
\end{code}

\section{Bungee Builder (monotonic stack)}
\begin{code}
\begin{minted}
void solve(){
   int n;cin>>n;
   vector<int> a(n);FOR(i,0,n)cin>>a[i];
   stack<pii> s;
   int ans=0;
   FOR(i,0,n){
       int bot=a[i];
       while(sz(s)){
           pii x = s.top();s.pop();
           bot = min(bot, x.S);
           if(x.F > a[i]){
               s.push({x.F,bot});
               ans=max(ans,a[i]-bot);
               break;
           }
           ans=max(ans,x.first-bot);
       }
       s.push({a[i],a[i]});
   }
   cout<<ans<<endl;
}
\end{minted}
\end{code}

\section{Arachnophobia (dijkstra + binary search)}
\begin{code}
\begin{minted}
int n,m,t,st,ed,k;
struct Edge{
   int v;ll w;
   bool operator<(const Edge &cmp) const {
       return cmp.w < w;
   }
};

vector<Edge> g[100005];
ll dis[100005],spiderdis[100005];

bool check(ll val){
   if(spiderdis[st] < val)return 0;
   if(spiderdis[ed] < val)return 0;
   FOR(i,0,n)dis[i]=Inf;
   priority_queue<Edge> pq;
   pq.push({st,0});
   while(pq.size()){
       auto node = pq.top(); pq.pop();
       if(dis[node.v] <= node.w)continue;
       dis[node.v] = node.w;
       for(auto it:g[node.v]){
           if(spiderdis[it.v] >= val && dis[it.v] > it.w + node.w){
               pq.push({it.v, it.w + node.w});
           }
       }
   }
   //FOR(i,0,n)cout<<dis[i]<<endl;
   return dis[ed] <= t;
}

void solve(){
   cin>>n>>m>>t;
   while(m--){
       int u,v,w;
       cin>>u>>v>>w;
       g[u].pb({v,w});
       g[v].pb({u,w});
   }
   cin>>st>>ed;
   cin>>k;
   priority_queue<Edge> pq;
   while(k--){
       int x;cin>>x;
       pq.push({x,0});
   }
   FOR(i,0,n)spiderdis[i]=Inf;
   while(pq.size()){
       auto node = pq.top(); pq.pop();
       if(spiderdis[node.v] <= node.w)continue;
       spiderdis[node.v] = node.w;
       for(auto it:g[node.v]){
           if(spiderdis[it.v] > it.w + node.w){
               pq.push({it.v,it.w + node.w});
           }
       }
   }
   ll ans=0,step= 1;
   FOR(i,0,62)step*=2;
   while(step > 0){
       if(check(ans+step)) ans += step;
       step=(step>>1);
   }
   cout<<ans<<endl;
}
\end{minted}
\end{code}

\section{Ascending Photo (Do some transition and DP)}
\begin{code}
\begin{minted}
void solve(){
    int n,cnt=0;cin>>n;
    vector<int> h;
    set<int> st;
    map<int,int> mp;
    FOR(i,0,n){
        int x;cin>>x;
        st.insert(x);
        if(h.empty() || h.back()!=x)h.pb(x);
    }
    for(auto it:st){
        if(!mp.count(it))mp[it] = cnt++;
    }
    n=sz(h);
    vector<vector<int>> pos(cnt);
    FOR(i,0,n){
        h[i] = mp[h[i]];
        pos[h[i]].pb(i);
    }
    pii best[2] = {{0,n}, {0,n}};
    FOR(i,0,cnt-1){
        pii nbest[2] = {best[0], best[1]};
        FOR(j,0,sz(pos[i])){
            int p = pos[i][j];
            if(p == n-1 || h[p]+1!=h[p+1])continue;
            pii s(0,n);
            if(p != best[0].second) s = best[0];
            else s = best[1];
            s.first++;
            s.second = p+1;
            if(pos[i+1].size() == 1) s.second = n;
            if(s > nbest[0]){
                nbest[1] = nbest[0];
                nbest[0] = s;
            }
            else if(s > nbest[1]) nbest[1] = s;
        }
        best[0] = nbest[0];
        best[1] = nbest[1];
    }
    cout<<n-1-best[0].first<<endl;
}
\end{minted}
\end{code}
\section{Pokemongogo (TSP)}
\begin{code}
\begin{minted}
int n,ans,cnt=0 , g[22][22]={}, dp[1<<22][22];
map<string,int> mp;
vector<int> pok[22];
vector<pii> p, v;

int dfs(int i,int j){
    if(dp[i][j] != -1)return dp[i][j];
    if(i == (1 << (n+1))-1 && j==0)return dp[i][j]=0;
    int res = 1e9+5;
    for(int k=0;k<=n;k++){
        if(!((i >> k) & 1)){
            res= min(res, dfs((i | (1 << k)), k) + g[j][k]);
        }
    }
    /*
     int b = i;
     while(b){
         cout<<(b&1);
         b/=2;
     }
     cout<<' '<<j<<' ';
     cout<<res<<endl;
    */
    
    return dp[i][j] = res;
}

void f(vector<int> a, int id){
    if(a.size() == n+1){
        
        for(int i=0;i<=n;i++){
            if(a[i])continue;
            a[i] = 1;
            
            int cn=0;
            for(int i=0, k=1;i<sz(a);i++, k*=2)
                cn += a[i] * k;
            if(dp[cn][i]!=-1){
                ans = min(ans, dp[cn][i] + g[0][i]);
                //for(int i=0;i<sz(a);i++)cout<<a[i];
                //cout<<' '<<i<<' ';
                //cout<<dp[cn][i]<<' ';
                //cout<<endl;
            }
            a[i]=0;
        }
        //cout<<endl;
        
    }
    int k = sz(a);
    FOR(i,0,sz(pok[id]))a.pb(1);
    for(int i=0;i < pok[id].size();i++){
        a[i+k]=0;
        f(a, id+1);
        a[i+k]=1;
    }
}

void solve(){
    cin>>n;
    p.resize(n+1);
    v.resize(n);
    FOR(i,0,n){
        int r,c;cin>>r>>c;
        string s;cin>>s;
        if(mp.count(s)==0){
            mp[s] = cnt++;
        }
        pok[mp[s]].pb(i);
        v[i] = {r,c};
    }
    p[0] = {0,0};
    for(int i=0,k=1;i<cnt;i++){
        for(int j=0;j<pok[i].size();j++){
            p[k++] = v[pok[i][j]];
        }
    }
    FOR(i,0,n+1){
        FOR(j,0,n+1){
            g[i][j] = abs(p[i].F-p[j].F) + abs(p[i].S-p[j].S);
            //cout<<g[i][j]<<' ';
        }
        //cout<<endl;
    }
    memset(dp, -1, sizeof(dp));
    dfs(0,0);
    ans=1e9+5;
    f({0}, 0);
    cout<<ans<<endl;
}
\end{minted}
\end{code}

\section{British Menu (Dp + SCC)}
\begin{code}
\begin{minted}
#include <cmath>
#include <cstdio>
#include <cstring>
#include <iostream>
#include <algorithm>
#include <vector>
#define MAX_V 200005
using namespace std;
int n,ans,ans1[MAX_V],ans2[MAX_V],c[MAX_V][6],cnum[MAX_V],id[MAX_V],dis[MAX_V][6],m;
vector<int> G[MAX_V];
vector<int> rG[MAX_V];
vector<int> vs;
bool used[MAX_V];
int cmp[MAX_V];
void add_edge(int from,int to)
{
    G[from].push_back(to);
    rG[to].push_back(from);
}
void dfs(int v)
{
    used[v]=1;
    for(int i=0;i<G[v].size();++i)
    {
        if(!used[G[v][i]])
            dfs(G[v][i]);
    }
    vs.push_back(v);
}
void rdfs(int v,int k)
{
    used[v]=1;
    cmp[v]=k;
    for(int i=0;i<rG[v].size();++i)
    {
        if(!used[rG[v][i]])
        rdfs(rG[v][i],k);
    }
}
int scc()
{
    memset(used,0,sizeof used);
    vs.clear();
    for(int v=1;v<=n;v++)
    {
        if(!used[v]) dfs(v);
    }
    memset(used,0,sizeof used);
    int k=1;
    for(int i=vs.size()-1;i>=0;--i)
    {
        if(!used[vs[i]])
        rdfs(vs[i],k++);
    }
    return k;
}
void cal(int now,int nowid,int f,int nowdis)
{
    used[now]=1;
    dis[now][nowid]=max(dis[now][nowid],nowdis);
    for(int i=0;i<G[now].size();++i)
    if(cmp[G[now][i]]==f&&!used[G[now][i]])
    {
        cal(G[now][i],nowid,f,nowdis+1);
    }
    used[now]=0;
    return;
}
int get2(int now);
int get1(int now)
{
    if(ans1[now]!=-1) return ans1[now];
    int nowans=1;
    for(int i=0;i<rG[now].size();i++)
    if(cmp[rG[now][i]]!=cmp[now])
    {
        nowans=max(nowans,get2(rG[now][i])+1);
    }
    return ans1[now]=nowans;
}
int get2(int now)
{
    if(ans2[now]!=-1) return ans2[now];
    int nowans=-1;
    for(int i=1;i<=cnum[cmp[now]];++i)
    {
        nowans=max(nowans,get1(c[cmp[now]][i])+dis[now][i]);
    }
    return ans2[now]=nowans;
}
int main()
{
    scanf("%d%d",&n,&m);
    int u,v;
    for(int i=1;i<=m;++i)
    {
        scanf("%d%d",&u,&v);
        add_edge(u,v);
    }
    scc();
    memset(used,0,sizeof used);
    for(int i=1;i<=n;++i)
    {
        ans1[i]=ans2[i]=-1;
        c[cmp[i]][++cnum[cmp[i]]]=i;
        id[i]=cnum[cmp[i]];
        cal(i,id[i],cmp[i],0);
    }
    for(int i=1;i<=n;++i)
    {
        ans=max(ans,get2(i));
    }
    cout<<ans;
    return 0;
}

\end{minted}
\end{code}

\section{Flight (Tree diameter/radius)}
\begin{code}
\begin{minted}
int n,vis[2505],dis[2505], siz[2505], h1[2505], h2[2505],c1[2505],c2[2505],p[2505];
vector<int> edge[2505];

pii treedia(int x){
    for(int i=1;i<=n;i++){
        vis[i]=0;
        dis[i]=0;
    }
    queue<int> q;
    q.push(x);
    while(q.size()){
        int top = q.front();
        q.pop();
        vis[top]=1;
        for(auto it:edge[top]){
            if(!vis[it]){
                dis[it] = dis[top]+1;
                q.push(it);
            }
        }
    }
    for(int i=1;i<=n;i++){
        if(dis[x] < dis[i])x = i;
    }
    for(int i=1;i<=n;i++){
        vis[i]=0;
        dis[i]=0;
    }
    q.push(x);
    while(q.size()){
        int top = q.front();
        q.pop();
        vis[top]=1;
        for(auto it:edge[top]){
            if(!vis[it]){
                dis[it] = dis[top]+1;
                q.push(it);
            }
        }
    }
    int id=0;
    for(int i=1;i<=n;i++){
        if(dis[id] < dis[i])id=i;
    }
    int ans=0;
    for(int i=1;i<=n;i++){
        if(dis[i] == dis[id]/2 + (dis[id]%2?1:0))ans = i;
    }
    if(dis[id] == 0)ans = x;
    return {ans, id};
}

void record(int x, int height, int child)
{
    if (height > h1[x])
    {
        h2[x] = h1[x]; c2[x] = c1[x];
        h1[x] = height; c1[x] = child;
    }
    else if (height > h2[x])
    {
        h2[x] = height; c2[x] = child;
    }
}

void dfs1(int x){
    h1[x] = h2[x] = 0;
    for(auto it:edge[x]){
        if(p[x] != it){
            p[it] = x;
            dfs1(it);
            record(x,h1[it]+1,it);
        }
    }
}

void dfs2(int x){
    if(p[x] != x){
        int y = p[x];
        if(c1[y] == x)record(x,h2[y]+1,y);
        else record(x,h1[y]+1,y);
    }
    for(auto it:edge[x]){
        if(it != p[x]){
            dfs2(it);
        }
    }
}

int treecentroid(int x){
    int ans=x;
    FOR(i,1,n+1) h1[i] = h2[i] = c1[i] = c2[i] = p[i] = 0;
    p[x] = x;
    dfs1(x);
    dfs2(x);
    for(int i=1;i<=n;i++){
        //cout<<h1[i]<<' ';
        if(h1[i] && h1[i] < h1[ans]){
            ans = i;
        }
    }
    //cout<<endl;
    return ans;
}

void solve(){
    cin>>n;
    vector<pii> ed;
    FOR(i,0,n-1){
        int u,v;cin>>u>>v;
        edge[u].pb(v);
        edge[v].pb(u);
        ed.pb({u,v});
    }
    int ans=1e9+5, a[4]={};
    for(auto [u,v]:ed){
        for(auto it = edge[u].begin();it!=edge[u].end();it++){
            if(*it == v){
                edge[u].erase(it);
                break;
            }
        }
        for(auto it = edge[v].begin();it!=edge[v].end();it++){
            if(*it == u){
                edge[v].erase(it);
                break;
            }
        }
        int k = treecentroid(u);
        int j = treecentroid(v);
        //cout<<k<<' '<<j<<endl;
        edge[k].pb(j);
        edge[j].pb(k);
        int y = treedia(u).second;
        if(ans > dis[y]){
            ans = dis[y];
            a[0] = u;
            a[1] = v;
            a[2] = k;
            a[3] = j;
        }
        for(auto it = edge[k].begin();it!=edge[k].end();it++){
            if(*it == j){
                edge[k].erase(it);
                break;
            }
        }
        for(auto it = edge[j].begin();it!=edge[j].end();it++){
            if(*it == k){
                edge[j].erase(it);
                break;
            }
        }
        edge[u].pb(v);
        edge[v].pb(u);
    }
    cout<<ans<<endl;
    cout<<a[0]<<' '<<a[1]<<endl;
    cout<<a[2]<<' '<<a[3]<<endl;
}
\end{minted}
\end{code}

\section{Rooted Subtree (Combinatorics + LCA)}
\begin{code}
\begin{minted}
vector<int> e[N];
int n,m,dep[N]={},siz[N]={},p[20][N]={};

void dfs(int x){
    siz[x] = 1;
    for(auto it:e[x]){
        if(p[0][x]!=it){
            p[0][it] = x;
            dep[it] = dep[x] + 1;
            dfs(it);
            siz[x] += siz[it];
        }
    }
}

int lca(int a,int b){
    if(dep[a] > dep[b])swap(a,b);
    int u=a,v=b;
    if(dep[a] != dep[b]){
        int dif = dep[b] - dep[a];
        for(int i=0;i<20;i++){
            if(dif&1)b = p[i][b];
            dif>>=1;
        }
    }
    if(a==b)return dep[v]-dep[u];
    for(int i=19;i>=0;i--){
        if(p[i][a] != p[i][b]){
            a = p[i][a];
            b = p[i][b];
        }
    }
    return dep[u]+dep[v]-2*dep[p[0][a]];
}

void solve(){
    cin>>n>>m;
    FOR(i,0,n-1){
        int u,v;
        cin>>u>>v;
        e[u].pb(v);
        e[v].pb(u);
    }
    p[0][1]=1;
    dfs(1);
    for(int i=1;i<20;i++){
        for(int j=1;j<=n;j++){
            p[i][j] = p[i-1][p[i-1][j]];
        }
    }
    while(m--){
        int u,v;cin>>u>>v;
        ll x = 1ll*lca(u,v)+1;
        cout<<x*(x+1)/2 + n - x<<endl;
    }
}
\end{minted}
\end{code}

\section{Stogovi (LCA + DSU)}
\begin{code}
\begin{minted}
vector<int> e[N];
int n,m,dep[N]={},siz[N]={},p[20][N]={},f[N]={};

void dfs(int x){
    siz[x] = 1;
    for(auto it:e[x]){
        dep[it] = dep[x] + 1;
        dfs(it);
        siz[x] += siz[it];
    }
}

int lca(int a,int b){
    if(dep[a] > dep[b])swap(a,b);
    if(dep[a] != dep[b]){
        int dif = dep[b] - dep[a];
        for(int i=0;i<20;i++){
            if(dif&1)b = p[i][b];
            dif>>=1;
        }
    }
    if(a==b)return a;
    for(int i=19;i>=0;i--){
        if(p[i][a] != p[i][b]){
            a = p[i][a];
            b = p[i][b];
        }
    }
    return p[0][a];
}

void prelca(){
    dfs(0);
    for(int i=1;i<20;i++){
        for(int j=1;j<=n;j++){
            p[i][j] = p[i-1][p[i-1][j]];
        }
    }
}

int find(int x){
    if(x == f[x])return x;
    return f[x] = find(f[x]);
}

void solve(){
    cin>>n;
    FOR(i,1,n+1)f[i]=i;
    vector<pii> ans;
    FOR(i,1,n+1){
        char c;
        cin>>c;
        int v,w;
        if(c=='a'){
            cin>>v;
            v = find(v);
            p[0][i] = v;
            e[v].pb(i);
        }
        else if(c=='b'){
            cin>>v;
            v = find(v);
            f[i] = p[0][v];
            ans.pb({v,-1});
        }
        else {
            cin>>v>>w;
            v = find(v);
            w = find(w);
            f[i] = v;
            ans.pb({v,w});
        }
    }
    prelca();
    for(auto [v,w]:ans){
        if(w == -1){
            cout<<v<<endl;
        }
        else{
            int x = lca(v,w);
            cout<<dep[x]<<endl;
        }
    }
}
\end{minted}
\end{code}

\section{LCA}
\begin{code}
\begin{minted}
// :80 <enter>
#include <bits/stdc++.h>
using namespace std;
typedef long long ll;
typedef pair<int, int> pii;
typedef pair<ll, ll> pll;
typedef pair<int, ll> pil;
typedef pair<ll, int> pli;
typedef pair<double,double> pdd;
#define SQ(i) ((i)*(i))
#define MEM(a, b) memset(a, (b), sizeof(a))
#define SZ(i) int(i.size())
#define FOR(i, j, k, in) for (int i=j ; i<(k) ; i+=in)
#define RFOR(i, j, k, in) for (int i=j ; i>=(k) ; i-=in)
#define REP(i, j) FOR(i, 0, j, 1)
#define REP1(i,j) FOR(i, 1, j+1, 1)
#define RREP(i, j) RFOR(i, j, 0, 1)
#define ALL(_a) _a.begin(),_a.end()
#define mp make_pair
#define pb push_back
#define eb emplace_back
#define X first
#define Y second
#ifdef jayinnn
#define TIME(i) Timer i(#i)
#define debug(...) do{\
    fprintf(stderr,"%s - %d (%s) = ",__PRETTY_FUNCTION__,__LINE__,#__VA_ARGS__);\
    _do(__VA_ARGS__);\
}while(0)
template<typename T>void _do(T &&_x){cerr<<_x<<endl;}
template<typename T,typename ...S> void _do(T &&_x,S &&..._t){cerr<<_x<<", ";_do(_t...);}
template<typename _a,typename _b> ostream& operator << (ostream &_s,const pair<_a,_b> &_p){return _s<<"("<<_p.X<<","<<_p.Y<<")";}
template<typename It> ostream& _OUTC(ostream &_s,It _ita,It _itb)
{
    _s<<"{";
    for(It _it=_ita;_it!=_itb;_it++)
    {
        _s<<(_it==_ita?"":",")<<*_it;
    }
    _s<<"}";
    return _s;
}
template<typename _a> ostream &operator << (ostream &_s,vector<_a> &_c){return _OUTC(_s,ALL(_c));}
template<typename _a> ostream &operator << (ostream &_s,set<_a> &_c){return _OUTC(_s,ALL(_c));}
template<typename _a> ostream &operator << (ostream &_s,deque<_a> &_c){return _OUTC(_s,ALL(_c));}
template<typename _a,typename _b> ostream &operator << (ostream &_s,map<_a,_b> &_c){return _OUTC(_s,ALL(_c));}
template<typename _t> void pary(_t _a,_t _b){_OUTC(cerr,_a,_b);cerr<<endl;}
#define IOS()
class Timer {
private:
    string scope_name;
    chrono::high_resolution_clock::time_point start_time;
public:
    Timer (string name) : scope_name(name) {
        start_time = chrono::high_resolution_clock::now();
    }
    ~Timer () {
        auto stop_time = chrono::high_resolution_clock::now();
        auto length = chrono::duration_cast<chrono::microseconds>(stop_time - start_time).count();
        double mlength = double(length) * 0.001;
        debug(scope_name, mlength);
    }
};
#else
#define TIME(i)
#define debug(...)
#define pary(...)
#define endl '\n'
#define IOS() ios_base::sync_with_stdio(0);cin.tie(0)
#endif

const ll MOD = 1000000007;
const ll INF = 0x3f3f3f3f3f3f3f3f;
const int iNF = 0x3f3f3f3f;
const ll MAXN = 100005;

mt19937 rng(chrono::steady_clock::now().time_since_epoch().count());

vector<vector<int>> adj;
vector<vector<int>> p;
vector<int> d;

void dfs(int cur, int par, int dep){
    d[cur] = dep;
    p[cur][0] = par;
    for(auto i: adj[cur]){
        if(i != par){
            dfs(i, cur, dep+1);
        }
    }
}

int lca(int x, int y){
    if(d[x] > d[y]) swap(x, y);
    if(d[x] != d[y]){
        int diff = d[y] - d[x];
        for(int i=0;i<20;i++){
            if(diff & 1) y = p[y][i];
            diff >>= 1;
        }
    }
    if(x == y) return x;
    for(int i=19;i>=0;i--){
        if(p[x][i] != p[y][i]){
            x = p[x][i];
            y = p[y][i];
        }
    }

    return p[x][0];
}

void solve(){
    int n;
    cin >> n;
    adj.resize(n+1);
    p.resize(n+1, vector<int>(20, 0));
    d.resize(n+1);
    int x, y;
    for(int i=0;i<n-1;i++){
        cin >> x >> y;
        adj[x].pb(y);
        adj[y].pb(x);
    }
    dfs(1, 0, 1);
    for(int i=1;i<20;i++){
        for(int j=1;j<=n;j++){
            p[j][i] = p[p[j][i-1]][i-1];
        }
    }
    ll ans = 0;
    debug(d);
    for(int i=1;i<=n;i++){
        for(int j=2*i;j<=n;j+=i){
            debug(i, j, lca(i, j));
            ll dis = d[i] - 2 * d[lca(i, j)] + d[j] + 1;
            debug(dis);
            ans += dis;
        }
    }
    cout << ans << endl;
}

/********** Good Luck :) **********/
int main () {
    TIME(main);
    IOS();
    int t = 1;
    while(t--){
        solve();
    }

    return 0;
}
\end{minted}
\end{code}

\section{Midterm pC}
\begin{code}
\begin{minted}
#include <bits/stdc++.h>
#define endl '\n'

using namespace std;
typedef long long ll;

vector<ll> pre(1e6+5, 0);

ll sum(int l, int r){
    if(l > r) return 0;
    else return pre[r] - pre[l-1];
}

int main(){
    ios_base::sync_with_stdio(0);
    cin.tie(0);
    int n;
    cin >> n;
    vector<ll> h(n+1);
    stack<pair<ll, ll>> s;
    for(int i=1;i<=n;i++){
        cin >> h[i];
    }
    for(int i=1;i<=n;i++){
        cin >> pre[i];
        pre[i] += pre[i-1];
    }
    
    ll ans = 0;
    for(int i=1;i<=n;i++){
        while(!s.empty() && s.top().first > h[i]){
            ll x = s.top().first;
            s.pop();
            ll idx;
            if(s.empty()){
                idx = 0;
            } else {
                idx = s.top().second;
            }
            ans = max(ans, x*2LL/3LL*(sum(idx+1LL, i-1)));
        }
        s.push({h[i], i});
    }
    while(!s.empty()){
        ll x = s.top().first;
        s.pop();
        ll idx;
        if(s.empty()){
            idx = 0;
        } else {
            idx = s.top().second;
        }
        ans = max(ans, x*2LL/3LL*(sum(idx+1LL, n)));
    }
    cout << ans << endl;
}
\end{minted}
\end{code}
\section{Midterm pE}
\begin{code}
\begin{minted}
#include <bits/stdc++.h>
#define endl '\n'
using namespace std;

typedef long long ll;

vector<int> par; // index
vector<ll> sz;
vector<ll> pre;

ll sum(int l, int r){
    if(l == 0) return pre[r];
    else return pre[r] - pre[l-1];
}

int find(int x){
    if(par[x] == x) return x;
    else return par[x] = find(par[x]);
}

void merge(int a, int b){
    int fa = find(a);
    int fb = find(b);
    if(fa == fb) return;
    if(sz[fa] < sz[fb]) swap(fa, fb);
    par[fb] = fa;
    sz[fa] += sz[fb];
}


int main(){
    ios_base::sync_with_stdio(0);
    cin.tie(0);
    int n, q;
    cin >> n >> q;
    vector<ll> ans;
    vector<ll> del_order;
    vector<int> query; // -1 for del, 0 for all, 1 for max
    vector<bool> run(n, true);
    string tmp;
    int x;
    
    par.resize(n);
    pre.resize(n);
    sz.resize(n, 1);
    par[0] = 0;
    cin >> pre[0];
    for(int i=1;i<n;i++){
        par[i] = i;
        cin >> pre[i];
        pre[i] += pre[i-1];
    }
    for(int i=0;i<q;i++){
        cin >> tmp;
        if(tmp == "D"){
            cin >> x;
            del_order.push_back(x);
            run[x] = false;
            query.push_back(-1);
        } else if(tmp == "QS"){
            query.push_back(0);
        } else {
            query.push_back(1);
        }
    }
    int l = 0;
    int r = 0;
    bool flag = run[0];
    ll cur_max = 0;
    ll total = 0;
    for(int i=1;i<n;i++){
        if(flag == false && run[i]){
            l = i;
            flag = true;
        } else if(run[i] && run[i-1]){
            merge(i, i-1);
            if(!flag){
                l = i;
                flag = true;
            }
        } else if(flag && run[i] == false){
            cur_max = max(cur_max, (i-1-l+1) * sum(l, i-1));
            total += (i-1-l+1) * sum(l, i-1);
            flag = false;
        }
    }
  
    if(flag){
        cur_max = max(cur_max, (ll)(n-1-l+1) * sum(l, n-1));
        total += (ll)(n-1-l+1) * sum(l, n-1);
    }
    // for(int i=0;i<n;i++){
    //     cout << i << " " << par[i] << endl;
    // }
    int idx = 0;
    reverse(query.begin(), query.end());
    reverse(del_order.begin(), del_order.end());
    for(int i=0;i<q;i++){
        if(query[i] == 0){
            ans.push_back(total);
        } else if(query[i] == 1){
            ans.push_back(cur_max);
        } else {
            x = del_order[idx++];
            run[x] = true;
            // cout << "x: " << x << endl;
            if(x == 0 || x == n-1){
                if(x == 0){
                    if(run[x+1]){
                        r = x + sz[find(x+1)];
                        total = total - (sz[find(x+1)] * sum(x+1, r)) + (ll)(1 + sz[find(x+1)]) * sum(x, r) ;
                        cur_max = max(cur_max, (ll)(1 + sz[find(x+1)]) * sum(x, r));
                        merge(x, x+1);
                    } else {
                        total += sum(x, x);
                        cur_max = max(cur_max, sum(x, x));
                    }
                } else {
                    if(run[x-1]){
                        l = x - sz[find(x-1)];
                        total = total - (sz[find(x-1)] * sum(l, x-1)) + (ll)(1 + sz[find(x-1)]) * sum(l, x);
                        cur_max = max(cur_max, (ll)(1 + sz[find(x-1)]) * sum(l, x));
                        merge(x-1, x);
                    } else {
                        total += sum(x, x);
                        cur_max = max(cur_max, sum(x, x));
                    }
                }
            } else {
                // merge w/ left and right
                // cur_max = (sz[x-1] + 1 + sz[x+1]) * sum(l, r);
                if(run[x-1] && run[x+1]){
                    l = x - sz[find(x-1)];
                    r = x + sz[find(x+1)];
                    total = total - (sz[find(x-1)] * sum(l, x-1)) - (sz[find(x+1)] * sum(x+1, r)) + (ll)(sz[find(x-1)] + 1 + sz[find(x+1)]) * sum(l, r);
                    cur_max = max(cur_max, (ll)(sz[find(x-1)] + 1 + sz[find(x+1)]) * sum(l, r));
                    merge(x-1, x);
                    merge(x, x+1);
                } else if(run[x-1]){
                    l = x - sz[find(x-1)];
                    total = total - (sz[find(x-1)] * sum(l, x-1)) + (ll)(1 + sz[find(x-1)]) * sum(l, x);
                    cur_max = max(cur_max, (ll)(1 + sz[find(x-1)]) * sum(l, x));
                    merge(x-1, x);
                } else if(run[x+1]){
                    r = x + sz[find(x+1)];
                    total = total - (sz[find(x+1)] * sum(x+1, r)) + (ll)(1 + sz[find(x+1)]) * sum(x, r);
                    cur_max = max(cur_max, (ll)(1 + sz[find(x+1)]) * sum(x, r));
                    merge(x, x+1);
                } else {
                    total += sum(x, x);
                    cur_max = max(cur_max, sum(x, x));
                }
            }
        }
    }
    reverse(ans.begin(), ans.end());
    for(auto i: ans){
        cout << i << endl;
    }
    
}
\end{minted}
\end{code}

\end{document}
